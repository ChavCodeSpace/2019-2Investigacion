\documentclass{article}
\usepackage[utf8]{inputenc}

\title{Computación, hasta la actualidad}
\author{Chrsitian Gallego Chaverra}
\date{Marzo 2020}

\usepackage{natbib}
\usepackage{graphicx}

\begin{document}

\maketitle

\section{Introducción}
Durante toda la historia de la humanidad el desarrollo matemático a ido de la mano con el desarrollo computacional. Cada avance en uno es seguido inmediatamente por un avance en el otro. Cuando la humanidad desarrolló el concepto del sistema de conteo en base diez, el ábaco fue una herramienta para hacerlo más fácil. Cuando las computadoras electrónicas fueron construidas para resolver ecuaciones complejas, campos como la dinámica de fluidos, teoría de los números, y la física química floreció.
Es así como hemos llegado a este punto en la historia, donde encontramos que la tecnología hace parte de casi todas las tareas diarias de casi todo el mundo. La computación ha encontrado aplicaciones en casi todos los campos de estudio y la humanidad cada vez esta mas acostumbrada con la tecnología. Encontramos desde casas a ciudades inteligentes, múltiples aparatos tecnológicos que pueden estar en red para un mismo usuario, lo que impulsa las redes de área personal (PAN) \citep{pan}.
Tanto a sido el crecimiento de la computación, que hoy en día la humanidad busca dotar de inteligencia a las maquinas y así hacerlas autosuficientes y pensantes para desarrollar diferentes problemas; a esto se le conoce como aprendizaje de maquina o machine learning, que va de la mano con la inteligencia artificial, otro de los grandes proyectos actuales de la humanidad que ha demostrado grandes resultados en donde se ha aplicado. La inteligencia artificial podría considerarse el resultado de la computación moderna y el siguiente escalón para una nueva era de computación y tecnología para la humanidad, pues, la IA ha facilitado muchas tareas y ayudado a grandes proyectos que antes no tenían forma de solución, impulsando a compañías y empresas tecnológicas a crecer cada vez mas. \citep{ia}
  
\section{Desarrollo}
A ciencia cierta no se sabe desde que época empezó el ser humano a interactuar con los números, pero tuvo que haber sido muy temprano, pues desde que el primer simio sintiera curiosidad de saber alguna cosa del entorno, le tocaría contar. Los números son una parte esencial de la rutina de la vida misma, están en todas partes y los usamos en todos lados, ya sea para asuntos que tengan que ver con la matemática misma o no, asuntos de la vida diaria y, por supuesto, la tecnología. Pero, remontándonos a las primeras maquinas que se usaron para hacer operaciones matemáticas, podríamos referirnos al ábaco \citep{abaco}, que se considera la primera máquina que podía realizar cálculos matemáticos de la historia.
Después de descubrir como se operaban los números y como se podían aplicar los resultados, la humanidad comenzó a avanzar. Aplicando matemáticas pudieron hacer mediciones y crear maquinas. La humanidad empezó a dar pensadores que revolucionaron la ciencia, se puede resaltar las contribuciones de Galileo\citep{galileo}, Kepler\citep{kepler}, Leibniz\citep{lv}, y sobre todo de Newton \citep{nwtnr} 
El calculo infinitesimal y la geometría analítica fueron fundamentales para el desarrollo de la matemática en los siglos XVII y XVIII, ya que la física empezaba a plantear innumerables problemas. El análisis matemático se encargo de ordenar estas ideas.
El siglo XIX es otro siglo muy importante en el desarrollo matemático. La aparición de personas como Gauss\citep{gauss}, Abel\citep{abel}, Galois\citep{galois}, Cauchy\citep{cauchy}, Riemann\citep{riemann}, entre otros, fue decisivo para crear nuevas ideas, con métodos y concepciones cada vez mas universales.
En esta época también surge el álgebra de Boole, de George Boole\citep{boole}, la cual es un aporte con proyecciones a futuro, pues es la base para el desarrollo de la microelectrónica, la cual es base de mucha tecnología actual.
Todos los conocimientos se fueron heredando por generaciones para que se aplicaran en los campos que iban surgiendo conforme avanzaba el tiempo y a la humanidad se le presentaban nuevos desafíos, nuevos problemas. También se iban integrando poco a poco todos los conocimientos adquiridos en todas las áreas de la ciencia y así es como el desarrollo matemático deriva en cada avance que haga la humanidad en cualquier ámbito científico.

\section{Conclusión}
La herencia del conocimiento ha sido una practica milenaria de la humanidad y la manera mas confiable de preservar los saberes que sirven de base para nuevas invenciones y nuevos descubrimientos. Es labor de cualquier que se dedique a la academia el preservar es compartir el conocimiento en lo que mas pueda; Ya sea dedicándose de lleno a la academia, enseñando o investigando; o ejerciendo en el ámbito que se haya estudiado para ayudar en el avance de la humanidad.
También es compromiso de la humanidad el no parar de investigar, no parar de crecer y enriquecerse científicamente. En este momento la inteligencia artificial marca un punto de salida para una nueva generación científica, este es un claro ejemplo del avance y la unión tecnológica que tiene la raza humana.

\bibliographystyle{plain}
\bibliography{references}
\end{document}
